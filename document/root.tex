\documentclass[11pt,a4paper]{article}
\usepackage{isabelle,isabellesym}
% further packages required for unusual symbols (see also
% isabellesym.sty), use only when needed

\usepackage{amssymb}
  %for \<leadsto>, \<box>, \<diamond>, \<sqsupset>, \<mho>, \<Join>,
  %\<lhd>, \<lesssim>, \<greatersim>, \<lessapprox>, \<greaterapprox>,
  %\<triangleq>, \<yen>, \<lozenge>

%\usepackage{eurosym}
  %for \<euro>

%\usepackage[only,bigsqcap]{stmaryrd}
  %for \<Sqinter>

%\usepackage{eufrak}
  %for \<AA> ... \<ZZ>, \<aa> ... \<zz> (also included in amssymb)

%\usepackage{textcomp}
  %for \<onequarter>, \<onehalf>, \<threequarters>, \<degree>, \<cent>,
  %\<currency>

% this should be the last package used
\usepackage{pdfsetup}

% urls in roman style, theory text in math-similar italics
\urlstyle{rm}
\isabellestyle{it}

% for uniform font size
%\renewcommand{\isastyle}{\isastyleminor}


\begin{document}

\title{Hybrid Multi-Lane Spatial Logic}
\author{Sven Linker}
\maketitle

\begin{abstract}
We present a semantic embedding of a spatio-temporal multi-modal
logic, specifically defined to reason about motorway traffic, into Isabelle/HOL. 
The semantic model is an abstraction of a motorway, 
emphasising local spatial properties, and parameterised by the types of sensors
deployed in the vehicles. We use the logic
to define controller constraints to ensure safety, i.e., the
absence of collisions on the motorway. 
After proving safety with a restrictive definition of
sensors, we relax these assumptions and show how to amend
the controller constraints to still guarantee safety.

Published in iFM 2017 \cite{Linker2017}.
\end{abstract}

Formal verification of autonomous vehicles on motorways is a challenging problem, 
due to the complex interactions between
dynamical behaviours and controller choices of the vehicles. 
 To overcome the complexities of proving safety properties, we proposed
to separate the dynamical behaviour from 
the concrete changes in space \cite{Hilscher2011}. To that end, we
defined
\emph{Multi-Lane Spatial Logic} (MLSL), which  was used to express guards and invariants
of controller automata defining a protocol for safe lane-change manoeuvres. Under the assumption
that all vehicles adhere to this protocol, we 
proved
that collisions were avoided. 
Subsequently, we presented an extension of MLSL to 
reason about 
changes in space over time, 
a system of natural deduction,  
and formally proved 
a safety theorem \cite{Linker2015a,Linker2015b}.
This proof was carried out manually and 
dependent on 
 strong
assumptions about the vehicles' sensors. 

We define
a semantic embedding of a further extension of MLSL, inspired 
by Hybrid Logic \cite{Brauner2010}. 
 Subsequently, we show how 
 the safety theorem can be proved within this embedding.
Finally, we alter this formal embedding by relaxing the assumptions on the sensors. We show
that the previously proven safety theorem does \emph{not} ensure
safety in this case, and how the controller constraints can 
be strengthened to guarantee safety.

\tableofcontents

% sane default for proof documents
\parindent 0pt\parskip 0.5ex

% generated text of all theories
\input{session}

% optional bibliography
\bibliographystyle{abbrv}
\bibliography{root}

\end{document}

%%% Local Variables:
%%% mode: latex
%%% TeX-master: t
%%% End:
