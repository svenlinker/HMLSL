\documentclass[11pt,a4paper]{article}
\usepackage{isabelle,isabellesym}
% further packages required for unusual symbols (see also
% isabellesym.sty), use only when needed

\usepackage{amssymb}
  %for \<leadsto>, \<box>, \<diamond>, \<sqsupset>, \<mho>, \<Join>,
  %\<lhd>, \<lesssim>, \<greatersim>, \<lessapprox>, \<greaterapprox>,
  %\<triangleq>, \<yen>, \<lozenge>

%\usepackage{eurosym}
  %for \<euro>

%\usepackage[only,bigsqcap]{stmaryrd}
  %for \<Sqinter>

%\usepackage{eufrak}
  %for \<AA> ... \<ZZ>, \<aa> ... \<zz> (also included in amssymb)

%\usepackage{textcomp}
  %for \<onequarter>, \<onehalf>, \<threequarters>, \<degree>, \<cent>,
  %\<currency>

% this should be the last package used
\usepackage{pdfsetup}

% urls in roman style, theory text in math-similar italics
\urlstyle{rm}
\isabellestyle{it}

% for uniform font size
%\renewcommand{\isastyle}{\isastyleminor}


\begin{document}

\title{HMLSL}
\author{By cthulhu}
\maketitle

\begin{abstract}
Formal verification of autonomous vehicles on motorways is a challenging problem, 
due to the complex interactions between
dynamical behaviours and controller choices of the vehicles. In previous 
work, we showed how an abstraction of motorway traffic, with an
emphasis on spatial properties, can be beneficial. In this paper,
we present a semantic embedding of a spatio-temporal multi-modal
logic, specifically defined to reason about motorway traffic, into Isabelle/HOL. 
The semantic model is an abstraction of a motorway, 
emphasising local spatial properties, and parameterised by the types of sensors
deployed in the vehicles. We use the logic
to define controller constraints to ensure safety, i.e., the
absence of collisions on the motorway. 
After proving safety with a restrictive definition of
sensors, we relax these assumptions and show how to amend
the controller constraints to still guarantee safety.

Published in iFM 2017 \cite{Linker2017}.
\end{abstract}

\tableofcontents

% sane default for proof documents
\parindent 0pt\parskip 0.5ex

% generated text of all theories
\input{session}

% optional bibliography
\bibliographystyle{abbrv}
\bibliography{root}

\end{document}

%%% Local Variables:
%%% mode: latex
%%% TeX-master: t
%%% End:
